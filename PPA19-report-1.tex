\documentclass[12pt,letterpaper,english,bibliography=totocnumbered]{scrartcl}

\usepackage{indentfirst}
\usepackage{appendix}
%\usepackage{fullpage}
%\usepackage{subfiles}
\usepackage[T1]{fontenc}
\usepackage[latin9]{inputenc}
\usepackage{color}
\usepackage{babel}
\usepackage{verbatim}
\usepackage[unicode=true,pdfusetitle,
bookmarks=true,bookmarksnumbered=false,bookmarksopen=false,
breaklinks=true,pdfborder={0 0 0},pdfborderstyle={},backref=false,colorlinks=true]
{hyperref}
\hypersetup{linkcolor=blue,citecolor=blue,urlcolor=blue}

\usepackage{booktabs}
\usepackage{multirow}
\usepackage{adjustbox}
\usepackage{threeparttable}
\usepackage[table]{xcolor}
\usepackage{csquotes}

\usepackage[backend=biber, style=authoryear,maxbibnames=99, dashed=false]{biblatex}
\addbibresource{mylibrary.bib}

% Prevent page breaks within paragraphs
% https://tex.stackexchange.com/questions/21983/how-to-avoid-page-breaks-inside-paragraphs
\widowpenalties 1 10000








%\documentclass[12pt,letterpaper,english]{scrartcl}
%\usepackage[T1]{fontenc}
%\usepackage[latin9]{inputenc}
%\usepackage{color}
%\usepackage{babel}
%\usepackage{verbatim}
%\usepackage{indentfirst}
%\usepackage{booktabs}
%
%\usepackage[unicode=true,pdfusetitle,
% bookmarks=true,bookmarksnumbered=false,bookmarksopen=false,
% breaklinks=true,pdfborder={0 0 0},pdfborderstyle={},backref=false,colorlinks=true]{hyperref}
% 
%\hypersetup{linkcolor=blue, citecolor=blue, urlcolor=blue}
%
%\usepackage{pdfpages}
%
%\usepackage[backend=biber, style=authoryear,maxbibnames=99, dashed=false]{biblatex}
%\addbibresource{mylibrary.bib}

%\makeatletter

%%%%%%%%%%%%%%%%%%%%%%%%%%%%%% LyX specific LaTeX commands.
%\pdfpageheight\paperheight
%\pdfpagewidth\paperwidth


%%%%%%%%%%%%%%%%%%%%%%%%%%%%%% User specified LaTeX commands.
%\usepackage{mciteplus}

%\makeatother

\begin{document}


\title{Coconut Rhinoceros Beetle Biological Control}
\author{Aubrey Moore, University of Guam}
\maketitle
\begin{description}	
	\item[Report ID:] AP19PPQS\&T00C168-PE-SA1-20
	\item[Report Type:] Performance Report
	\item[Reporting Period:]  SA1
	\item[Performance Period:] August 8, 2019 - February 8, 2010
\item[Federal Award Identification Number:] AP19PPQS\&T00C168
\item[Agreement Title:] PPA7721 6R.0117.00 Guam CRB BC
\end{description}

\begin{footnotesize}
\url{https://github.com/aubreymoore/FY19-PPA-Report-1/raw/master/PPA19-report-1.pdf}
\end{footnotesize}

\newpage
\section*{In a Nut Shell}

\definecolor{amlightgray}{rgb}{0.95, 0.95, 0.95}
\fcolorbox{black}{amlightgray}{
	\begin{minipage}{\textwidth}
		\begin{itemize}
			
			\item The primary objective of this project was to find an isolate of \textit{Oryctes rhinoceros} nudivirus (OrNV) which can be used as an effective biological control agent for CRB-G biotype of coconut rhinoceros beetle (CRB). Laboratory bioassays of four isolates indicated that one of them, OrNV isolate V23B, is pathogenic for CRB-G and can be considered as a potential biocontrol agent for this pest.
			
			\item A secondary objective of this project was to develop a CRB damage monitoring system. A digital image analysis system was developed to detect and quantify V-shaped cuts to fronds and coconut palm mortality caused by CRB. The heart of this system is an object detector, trained by deep learning technology, which locates CRB damage symptoms on frames from georeferenced roadside video surveys. This object detector can be used to automate detection, quantification and to map changes in CRB damage over time and space.
			
			\item Uncontrolled outbreaks of CRB-G is a major problem for Pacific islands. Outbreaks of this highly invasive biotype are damaging and killing palms in Guam, Rota, Hawaii, Palau, Papua New Guinea, and the Solomon Islands. Without effective control of these outbreaks, the problem will spread to other Pacific islands, resulting in a human tragedy when it reaches atolls were islanders still really on coconut palm as the \textit{tree of life}. Project resources, time and effort were used to facilitate communication among an \textit{ad hoc} collaboration of entomologists working on the CRB-G problem throughout the Pacific.
			
		\end{itemize}
	\end{minipage}
}


\newpage
\tableofcontents{}

\newpage{}

\section{Staffing}

Staff for this project currently comprises of the PI, a post-doc and
a technician. 

\begin{itemize}
	
\item Funding from the Department of Interior, Office of Island Affairs
was used to hire an insect pathologist for a 2 year term. Dr. James
Grasela was recruited and started work at UOG on June 24, 2018.

\item Mr. Chris Cayanan was hired as a technician for this project on DATE.

\end{itemize}

\section{Laboratory Bioassays of OrNV}




The major goal of this project is to find an effective biological
control agent for coconut rhinoceros beetle biotype G (CRB-G). 

Prior to arrival of CRB on Guam during 2007, coconut rhinoceros beetle
infestations of Pacific islands were readily controled by classical
biological control using Oryctes nudivirus (OrNV). Following a lack
of response to release of OrNV on Guam, research showed that the Guam
CRB population is a genetically distinct virus-resistant biotype which
has become known as CRB-G. This biotype is highly invasive and is
causing massive damage to coconut and oil palms in Papua New Guinea
and the Solomon Islands. CRB-G has also invaded Oahu and Rota. Eradication
attempts have been launched on these two islands. 

The current project is part of an informal international collaboration
among Pacific island entomologist working to find solutions for CRB-G
management. The current focus is on testing new OrNV isolates in the
hope of finding on or more that can be used as an effective biological
control agent for CRB-G.

\subsection{Haemocoel Injection Bioassays}

In this series of assays, we tested 4 islates of OrNV which had produced
by an insect cell culture at the AgResearch Laboratory in New Zealand.
Adult beetles were dosed by direct injection into the haemocoel. This
series is a preliminary test for pathogenicity. Insignificant differences
in mortality curves between virus treatment and the other two treatments
(control treatment and heat-inactivated virus) is an indicator of
pathogenicity. Gut tissue samples have been preserved for histology
and molecular analysis. 

The following is a brief summary of results. Details are provided
in the appended technical reports. Results indicate that isolates
DUG42 and MALB are not pathogenic for CRB-G, but isolates PNG and
V23B are pathogenic. Bioassays in which adult beetles are dosed \emph{per
os} are underway and results will be available in a future report.

\subsubsection{OrNV Isolate DUG42}

Origin: Philippines; 2 replicates; 30 beetles in total

No significant differences among mortality cuves. {[}control, heat-inactivated
virus, virus{]}

\subsubsection{OrNV Isolate MALB}

Origin: Malaysia; 2 replicates; 30 beetles in total

No significant differences among mortality curves. {[}control, heat-inactivated
virus, virus{]}

\subsubsection{OrNV Isolate PNG}

Origin: Papua New Guinea; 4 replicates; 71 beetles in total

Mortality curves in 2 significantly different groups: {[}control,
heat-inactivated virus{]}, {[}virus{]}

\subsubsection{OrNV Isolate V23B}

Origin: Solomon Islands; 4 repicates; 66 beetles in total

Mortality curves in 2 significantly different groups: {[}control,
heat-inactivated virus{]}, {[}virus{]}

% Please add the following required packages to your document preamble:
% \usepackage{booktabs}
% \usepackage{multirow}
\begin{table}[h]
	\begin{adjustbox}{width=\columnwidth,center}
		
		\begin{threeparttable} 
			\caption{\textit{Oryctes rhinoceros} nudivirus (OrNV) bioassay results summary.}
			\label{bioassay_results}
			
			
			\begin{tabular}{ l l l c c c c }
				\toprule
				OrNV                  & bioassay                                        & method\tnote{1} & beetles & replicates & virus                           & inactivated                     \\
				isolate               &                                                 &                 &         &            & mortality (\textit{p})\tnote{2} & virus                           \\
				&                                                 &                 &         &            &                                 & mortality (\textit{p})\tnote{3} \\ \bottomrule
				DUG42                 & DUG42\cite{moore_bioassay_2019}                 & injection       & 30      & 2          & 40\% (0.65)                     & 40\% (0.65)                     \\ \midrule
				\multirow{2}{*}{MALB} & MALB\cite{moore_bioassay_2019-6}                & injection       & 30      & 2          & 50\% (0.37)                     & \hphantom{0}0\% (1.00)          \\
				& MALBperOS\cite{moore_bioassay_2019-7}           & per os          & 13      & 1          & -60\% (1.00)                    & 20\% (1.00)                     \\ \midrule
				\multirow{2}{*}{PNG}  & PNG\cite{moore_bioassay_2019-2}                 & injection       & 81      & 4          & \cellcolor{yellow}{90\% (0.00)} & \hphantom{0}5\% (1.00)          \\
				& PNGperOS\cite{moore_bioassay_2019-9}            & per os          & 21      & 1          & \hphantom{0}0\% (1.00)          & \hphantom{0} 0\% (1.00)         \\ \midrule
				\multirow{4}{*}{V23B} & V23B\cite{moore_bioassay_2019-3}                & injection       & 66      & 4          & \cellcolor{yellow}{88\% (0.00)} & \hphantom{0}0\% (1.00)          \\
				& V23BperOS\cite{moore_bioassay_2019-5}           & per os          & 32      & 2          & 80\% (0.07)                     & 20\% (0.69)                     \\
				& V23-large\_bioassay\cite{moore_bioassay_2019-4} & per os          & 53      & 1          & \cellcolor{yellow}{42\% (0.00)} & -                               \\
				& V23\_perOSIN\cite{moore_bioassay_2019-1}        & per os          & 16      & 1          & 60\% (0.06)                     & -                               \\ \bottomrule
			\end{tabular}
			\begin{tablenotes}[para]
				\item[1] Adult beetles were dosed either by direct injection of virus suspension into the haemocoel or by applying a droplet containing virus to mouthparts. \\ 
				\item[2] Percent mortality in beetles treated with virus, adjusted for untreated control mortality; 
				number in parentheses is the \textit{p}-value resulting from a Fisher's exact test of significant difference between mortality of treated and untreated beetles. \\
				\item[3] Percent mortality in beetles treated with heat inactivated virus, adjusted for untreated control mortality; 
				number in parentheses is the \textit{p}-value resulting from a Fisher's exact test of significant difference between mortality of treated and untreated beetles. 
			\end{tablenotes}
			
		\end{threeparttable}
	\end{adjustbox}
\end{table}


\subsection{PCR}

\cite{grasela_technical_2020} 
\cite{grasela_technical_2020-1}
\cite{grasela_technical_2020-2}

\section{Environmental Cabinets and CRB Rearing}

Three environmental cabinets which allow control of temperature, relative
humidity, and lighting for insect rearing were procured and installed.
These chambers are set to maintain 30\textdegree{}C, 80\% RH and
12h photoperiod.

After a power outage caused by a typhoon, one of the cabinets malfunctioned.
It heated beyond the setpoint and killed all beetles. To prevent this
problem from recurring, controllers for all three units have been
programmed to send email to project staff whenever a fault is detected.

The project does not currently rear beetles form egg to adult. Because
CRB are so numerous on Guam, it is far more efficient to field collect
prepupae, pupae and adults and rear these to the age required for
bioassays. Adults are fed banana slices.

\section{CRB Damage Survey}
\begin{itemize}
\item A 360 degree digital camera and accessories were purchased.
\item A protocol for roadside CRB damage surveys using digital imagery was
developed and trial runs were made.
\item A workflow for scoring CRB damage from digital imagery is under development.
\end{itemize}

\section{Regional Collaboration}

An informal collaboration, the \textit{CRB-G Action Group}, has been formed among Pacific-based entomologists working on the CRB-G problem. Participants from Guam, Hawaii, Palau, Papua New Guinea, Solomon Islands, Fiji, Malaysia, Japan and New Zealand have met several times and future meetings are planned (Table \ref{tab:action-group}).  This is an \textit{ad hoc} group which has been organized by Dr. Trevor Jackson and Sean Marshall of AgResearch New Zealand. AgResearch is recognized as a global center for expertise on biological control of CRB. AgResearch scientists have worked on CRB in the south Pacific for several decades and they maintain a library of OrNV isolates in cell culture. The New Zealand government has recently committed several million dollars to aid response to CRB-G in the south Pacific islands.

Although individual institutions working to find a solution to the CRB-G problem on American-affiliated islands in the northern Pacific have secured funding from multiple, short-term grants, attempts to secure funding to support a sustainable well-coordinated regional project have been unsuccessful. We respectfully request SERDP funding to support collaboration and cooperation on progress towards solving the CRB-G problem among partners in the American Pacific in partnership with international colleagues in the south Pacific. If our request is granted the project PI and staff at NCSU would be tasked with organizing quarterly Pacific-wide teleconferences and helping to organize and fund participation from the American Pacific in annual CRB-G Action Group conferences in 2021 through 2024.

\begin{table}[h]
	\centering
	\caption{Meetings of the CRB-G Action Group}
	\begin{tabular}{l}
		\toprule
		2015 Pacific Entomology Conference, Honolulu, HI, USA \\
		2016 International Congress of Entomology, Orlando, USA \\
		2017 Japanese Society for Insect Pathology, Tokyo, Japan \\
		2018 Society for Invertebrate Pathology, Gold Coast, Australia \\
		2019 XIX International Plant Protection Congress, Hyderabad, India \\
		2020 (tentative): Pacific Plant Protection Organization, Guam \\
		\bottomrule
	\end{tabular}
	\label{tab:action-group}
\end{table}	

\subsection{Participation in Scientific Meetings}

Moore and Grasela participated at the MEETING in a symposium entitled The challenge of coconut rhinoceros beetle , Oryctes rhinoceros, to palm production and prospects for control in a changing world.

They also participated in a Coconut Rhinoceros Beetle Action Group meeting.

Presentation \cite{moore_status_2019}

Presentation \cite{marshall_challenge_2019}

\subsection{Development of Online Resources}

\begin{itemize}
	\item CRB Bibliography \url{https://github.com/aubreymoore/CRB-Bibliography}
	\item Interactive CRB Invasion History Map	\url{https://aubreymoore.github.io/crbdist/mymap.html}
	\item CRB Wiki Site \url{https://guaminsects.net/CRBG}
	\item CRB-G Facebook Site \url{https://www.facebook.com/groups/crbg07} 
\end{itemize}

\newpage

\section{Appendices}

\appendix

\section{\label{sec:Appendix-A}Appendix A: Technical Report: Injection Bioassay
of OrNV Isolate DUG42}

See following page.

%\includepdf[pages=-]{DUG42}

\section{\label{sec:Appendix-B}Appendix B: Technical Report: Injection Bioassay
of OrNV Isolate MALB}

See following page.

%\includepdf[pages=-]{MALB}

\section{\label{sec:Appendix-C}Appendix C: Technical Report: Injection Bioassay
of OrNV Isolate PNG}

See following page.

%\includepdf[pages=-]{PNG}


\section{\label{sec:Appendix-D}Appendix D: Technical Report: Injection Bioassay
of OrNV Isolate V23B}

See following page.

%\includepdf[pages=-]{V23B}
\clearpage
\printbibliography[heading=bibintoc]


\end{document}
