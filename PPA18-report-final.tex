\documentclass[12pt,
letterpaper,english,bibliography=totocnumbered, abstract=on]{scrartcl}

\usepackage{indentfirst}
\usepackage{appendix}
%\usepackage{fullpage}
%\usepackage{subfiles}
\usepackage[T1]{fontenc}
\usepackage[latin9]{inputenc}
\usepackage{color}
\usepackage{babel}
\usepackage{verbatim}
\usepackage[unicode=true,pdfusetitle,
bookmarks=true,bookmarksnumbered=false,bookmarksopen=false,
breaklinks=true,pdfborder={0 0 0},pdfborderstyle={},backref=false,colorlinks=true]
{hyperref}
\hypersetup{linkcolor=blue,citecolor=blue,urlcolor=blue}

\usepackage{booktabs}
\usepackage{multirow}
\usepackage{adjustbox}
\usepackage{threeparttable}
\usepackage[table]{xcolor}
\usepackage{csquotes}
\usepackage{soul} % for hiliting text: \hl

\usepackage[backend=biber, style=authoryear, maxbibnames=99, dashed=false]{biblatex}
\addbibresource{mylibrary.bib}
\addbibresource{CRB.bib}

% Prevent page breaks within paragraphs
% https://tex.stackexchange.com/questions/21983/how-to-avoid-page-breaks-inside-paragraphs
\widowpenalties 1 10000


\begin{document}
\titlehead{USDA-APHIS Final Report}
\title{Coconut Rhinoceros Beetle Biological Control}
\author{Aubrey Moore, University of Guam}
\maketitle
\begin{description}	
	\item[Report ID:] \hl{AP18PPQFO000C402-PE-SA1-20}
	\item[Report Type:] Final Report
	\item[Reporting Period:]  \hl{SA1-20}
	\item[Performance Period:] \hl{October 10, 2019 through March 10, 2020}
\end{description}

\begin{footnotesize}
\url{https://github.com/aubreymoore/FY19-PPA-Report-1/raw/master/PPA18-report-final.pdf}
\end{footnotesize}

\newpage{}

\begin{abstract}
	\textbf{CRB Biological Control.} The primary objective of this project is to find an isolate of \textit{Oryctes rhinoceros} nudivirus (OrNV) which can be used as an effective biological control agent for the CRB-G biotype of coconut rhinoceros beetle (CRB). 
	
	Laboratory tests indicate that OrNV from two sources can be considered as potential biocontrol agents CRB-G: OrNV isolate V23B maintained in insect tissue culture by AgResearch New Zealand and OrNV in bodies of CRB collected in Taiwan for the current project. Further laboratory testing of these virus samples is underway. 
	
	PCR tests of recently collected CRB-G adults on Guam indicate presence of OrNV in this population. This virus could be from OrNV autodissemination early in the current project or from fortuitous introduction.
	\vspace{0.25in}
	
	\textbf{CRB Damage Survey.} A secondary objective of this project is to develop a CRB damage monitoring system. 
	
	A digital image analysis system has been developed to detect and quantify V-shaped cuts to fronds and coconut palm mortality caused by CRB. The heart of this system is an object detector, trained by deep learning technology, which locates CRB damage symptoms on frames from georeferenced roadside video surveys. This object detector can be used to automate detection, quantification and to map changes in CRB damage over time and space and can also be used fro early detection of CRB invasion.
	
	A working prototype of the system has been built.
	\vspace{0.25in}

	\textbf{Regional Collaboration.} Uncontrolled outbreaks of CRB-G is a major problem for Pacific islands. Outbreaks of this highly invasive biotype are damaging and killing palms in Guam, Rota, Hawaii, Palau, Papua New Guinea, and the Solomon Islands. Without effective control of these outbreaks, the problem will spread to other Pacific islands, resulting in a human tragedy when it reaches atolls were islanders still rely on coconut palm as the \textit{tree of life}. 
	
	Project resources, time and effort were used to facilitate communication among an \textit{ad hoc} collaboration of entomologists working on the CRB-G problem throughout the Pacific. Project staff participated in a symposium and planning meeting of the CRB-G Action Group at the XIX International 
	Plant Protection Congress in November 2019.

\end{abstract}


\newpage
\tableofcontents{}

\newpage

\section{Background}

The major goal of this project was to find an effective biological
control agent for coconut rhinoceros beetle biotype G (CRB-G).

Prior to arrival of CRB-G on Guam during 2007, coconut rhinoceros beetle
infestations of Pacific islands were readily dealt with by classical
biological control using \textit{Oryctes} nudivirus (OrNV), a pathogen specific to rhinoceros beetles. 
Following a lack
of response to release of OrNV on Guam, research showed that the Guam
CRB population is a genetically distinct virus-resistant biotype which
has become known as CRB-G (\cite{marshall_new_2017-1}). This biotype is highly invasive and is
causing massive damage to coconut and oil palms after recent invasion of Papua New Guinea
and the Solomon Islands. CRB-G has also invaded Oahu and Rota. Eradication
attempts have been launched on these two islands.

Additional goals for this project are to establish a CRB damage survey to evaluate efficacy of biocontrol and other tactics, and to maintain and facilitate collaboration with other Pacific island entomologists working to find solutions for CRB-G management.

\section{Staffing}

Staff for this project currently comprises of the PI, a post-doc and
a technician. 

\begin{itemize}
	
\item Funding from the Department of Interior, Office of Island Affairs
was used to hire an insect pathologist for a 2 year term. Dr. James
Grasela was recruited and started work at UOG on June 24, 2018.

\item Funding from this USDA-APHIS project was used to hire Mr. Chris Cayanan as a technician. Mr. Cayanan was hired during December, 2019, as a replacement for Mr. Ian Iriarte who resigned to accept another job.

\end{itemize}

\section{Biological Control}

\subsection{Laboratory Bioassays of OrNV Isolates}

Four isolates of OrNV were evaluated as candidate biological control agents for CRB-G in a series of laboratory bioassays. Virus sample preparations came from Dr. Sean Marshall's lab at AgResearch New Zealand were they are maintained in insect cell culture.

\begin{description}
	\item[DUG42] Collected from Dumaguete, Negros Island, Philippines in 2017
	\item[MALB] Collected from Malaysia, details not available.
	\item[PNG] Collected from Rabaul, Papua New Guinea in 1988
	\item[V23B] Collected from southern Luzon, Philippines in 1980
\end{description}

During laboratory bioassays, we dosed CRB-G adults with samples of the OrNV islolates, and observed mortality and changes in mass for one month. Each beetle was kept in isolation and individual records were stored in a laboratory information system developed for this application (Section \ref{lims}).



Bioassay results, displayed in Table \ref{tab: bioassay results}, indicate that one of the isolates, V23B, is pathogenic for CRB-G when doses are applied by placing droplets of virus suspension on mouthparts of adult beetles.


% Please add the following required packages to your document preamble:
% \usepackage{booktabs}
% \usepackage{multirow}
\begin{table}[h]
	\begin{adjustbox}{width=\columnwidth,center}
		
		\begin{threeparttable} 
			\caption{\textit{Oryctes rhinoceros} nudivirus (OrNV) bioassay results summary.}
			\label{tab: bioassay results}
			
			
			\begin{tabular}{ l l l c c c c }
				\toprule
				OrNV                  & bioassay                                        & method\tnote{1} & beetles & replicates & virus                           & inactivated                     \\
				isolate               &                                                 &                 &         &            & mortality (\textit{p})\tnote{2} & virus                           \\
				&                                                 &                 &         &            &                                 & mortality (\textit{p})\tnote{3} \\ \bottomrule
				DUG42                 & DUG42 \parencite{moore_bioassay_2019}                 & injection       & 30      & 2          & 40\% (0.65)                     & 40\% (0.65)                     \\ \midrule
				\multirow{2}{*}{MALB} & MALB \parencite{moore_bioassay_2019-6}                & injection       & 30      & 2          & 50\% (0.37)                     & \hphantom{0}0\% (1.00)          \\
				& MALBperOS \parencite{moore_bioassay_2019-7}           & per os          & 13      & 1          & -60\% (1.00)                    & 20\% (1.00)                     \\ \midrule
				\multirow{2}{*}{PNG}  & PNG \parencite{moore_bioassay_2019-2}                 & injection       & 81      & 4          & \cellcolor{yellow}{90\% (0.00)} & \hphantom{0}5\% (1.00)          \\
				& PNGperOS \parencite{moore_bioassay_2019-9}            & per os          & 21      & 1          & \hphantom{0}0\% (1.00)          & \hphantom{0} 0\% (1.00)         \\ \midrule
				\multirow{4}{*}{V23B} & V23B \parencite{moore_bioassay_2019-3}                & injection       & 66      & 4          & \cellcolor{yellow}{88\% (0.00)} & \hphantom{0}0\% (1.00)          \\
				& V23BperOS \parencite{moore_bioassay_2019-5}           & per os          & 32      & 2          & 80\% (0.07)                     & 20\% (0.69)                     \\
				& V23-large\_bioassay \parencite{moore_bioassay_2019-4} & per os          & 53      & 1          & \cellcolor{yellow}{42\% (0.00)} & -                               \\
				& V23\_perOSIN \parencite{moore_bioassay_2019-1}        & per os          & 16      & 1          & 60\% (0.06)                     & -                               \\ \bottomrule
			\end{tabular}
			\begin{tablenotes}[para]
				\item[1] Adult beetles were dosed either by direct injection of virus suspension into the haemocoel or by applying a droplet containing virus to mouthparts. \\ 
				\item[2] Percent mortality in beetles treated with virus, adjusted for untreated control mortality; 
				number in parentheses is the \textit{p}-value resulting from a Fisher's exact test of significant difference between mortality of treated and untreated beetles. \\
				\item[3] Percent mortality in beetles treated with heat inactivated virus, adjusted for untreated control mortality; 
				number in parentheses is the \textit{p}-value resulting from a Fisher's exact test of significant difference between mortality of treated and untreated beetles. 
			\end{tablenotes}
			
		\end{threeparttable}
	\end{adjustbox}
\end{table}

In a separate experiment, macerated guts from OrNV infected beetles from field collections of CRB in Taiwan were fed to CRB-G field-collected on Guam. PCR results indicated that the Taiwanese OrNV propagated in the Guam beetles (See Subsection \ref{(subsec: pcr results)}).

We now have two isolates of OrNV which can be considered as candidate biocontrol agents for further testing: V23B and Taiwan. 

\subsection{Virus Transmission Experiment}
\label{sec: virus transmission expt}

This experiment was performed to determine if OrNV isolate V23B can be transmitted from a dosed CRB adult to an undosed CRB adult. If a virus is not contagious in the lab, it will probably have no potential as a biocontrol agent. 

Unfortunately, the experiment failed due to very high mortality in the experimental control group and results are inconclusive. For details, see \cite{grasela_guam_2020}. This experiment needs to be repeated using pathogen-free, laboratory-reared adults (See Section \ref{sec: rearing}).

\subsection{PCR Tests for OrNV Detection}
\label{sec: pcr}

Previously, our laboratory relied on outside collaboration for molecular testing to determine presence of OrNV in CRB tissues. We recently acquired access to equipment and supplies which allow us to use PCR (polymerase chain reaction) techniques for OrNV detection. We tested five different primer pairs for detection of OrNV in reference samples from AgResearch New Zealand and were successful with all five. Identity of DNA fragments was confirmed as OrNV using a commercial DNA sequencing service.

Details of our PCR technique and initial series of tests can be found in \cite{grasela_technical_2020}, 
\cite{grasela_technical_2020-1} and \cite{grasela_technical_2020-2}.

Results from PCR tests indicate:
\label{(subsec: pcr results)}

\begin{itemize}

\item OrNV is present in the Guam adult CRB population: 18\% (10 of 47) gut samples dissected from field collected CRB tested positive for OrNV.

\item  OrNV is present in the adult Taiwanese CRB population:  7\% (5 of 67) gut samples dissected from field collected CRB tested positive for OrNV. 

\item  OrNV from Taiwanese adult CRB propagates in Guam adult CRB: 12\% (5 of 41) Guam CRB adults dosed with Taiwanese gut preparations tested positive.   

\end{itemize}

\subsection{Survey to Determine Incidence of OrNV Infection in the Guam CRB Population}

A survey to determine incidence of OrNV infection in the Guam CRB population was performed following positive PCR tests for OrNV infection in 18\% (10 of 47) gut samples dissected from locally collected beetles. The experimental protocol developed to minimize false positive tests is detailed in (\cite{mooreExperimentalPlanDetermining2020}). Preliminary results (\cite{graselaInvestigationDeterminePresence2020}) indicate that previous PCR tests indicating OrNV infection
were most likely caused by laboratory contamination.

\clearpage
\subsection{CRB Rearing}
\label{sec: rearing}

Experimental beetles were field-collected on Guam as prepupae, pupae or adults from breeding sites or as adults from pheromone traps.  Each beetle was given a serial number and was reared individually in a Mason jar stored in one of three environmental chambers set for 30 degrees Celsius, 80\% relative humidity and 12 hour photoperiod. Each adult beetle was fed weekly with a slice of banana. Detailed information on the CRB rearing facility can be found in a document prepared in support of a USDA-APHIS permit to allow importation of live CRB for laboratory bioassays \parencite{moore_additional_2019}.

\subsubsection{Urgent Need for a Pathogen-free CRB Laboratory Colony} 

To date, CRB used in bioassays and other laboratory experiments were field collected on Guam, rather than from reared in a laboratory colony. This was done because a very high population density of rhino beetles on Guam made field collection far more efficient in terms of time and resources than laboratory rearing. However, field collected beetles are often infected with pathogens, especially the fungus, \textit{Anisopliae majus}, which was introduced on Guam as a classical biological control agent. Presence of these pathogens in experimental animals resulted in unpredictable and often high mortality rates in experimental control groups, leading to inconclusive results  such as experienced with the recent virus transmission experiment (\ref{sec: virus transmission expt}).

Recent discovery of OrNV within the wild Guam rhino beetle population (\ref{sec: pcr}) now makes it mandatory to establish a sterile CRB-G laboratory colony to supply experimental animals.

\subsubsection{Laboratory Information Management System}\label{lims}

We developed an online database which we use as a laboratory information management system for maintaining individual records for beetles in our rearing program \parencite{moore_coconut_2019-1}. This system was developed using the \href{http://www.web2py.com/}{web2py python web framework} and it is available on the web at \url{http://aubreymoore.pythonanywhere.com/rearing3}. 

\subsubsection{Acquisition of a Virus-Susceptible CRB Biotype for Comparative Bioassays}

Since discovery of the CRB-G biotype on Guam \parencite{marshall_new_2017-1}, we have been operating under the hypothesis that this biotype is significantly more tolerant (resistant) to pathogenic effects of OrNV isolates previously used as effective biocontrol agents for CRB invading Pacific Islands. It has also been hypothesized that CRB-G has different behavioral characteristics, such as a significantly reduced attraction to oryctalure. However, comparative laboratory bioassays have not been performed to test these hypotheses.

We applied for and have been granted a USDA-APHIS import permit for live CRB which will allow us to establish a laboratory colony of CRB from presumed non-virus-resistant populations (See \parencite{moore_additional_2019} and \parencite{usda-aphis_crb_2019}). We have designed custom shipping containers to facilitate secure transport of live CRB \parencite{moore_container_2017-1}.

We plan to import CRB from American Samoa with assistance from our collaborator, Dr. Mark Schmaedick, American Samoa Community College. Plans for project staff to visit American Samoa in December 2019 failed because of travel restrictions prompted by a measles outbreak and in March 2020 because of travel restrictions prompted by the corona virus pandemic. 

\section{CRB Damage Survey}

The objective of this component of the project was to develop an automated system to evaluate CRB damage by image analysis of roadside video surveys.  We completed a \textit{proof of concept} trial in which deep learning algorithms were used to train an object detector which locates and counts dead and CRB-damaged coconut palms in video streams.  Visual results are presented in a YouTube post \parencite{moore_training_2019} and technical details are available in an Open Science Framework Project \parencite{moore_open_2019}.

\section{Regional Collaboration}

An informal collaboration, the \textit{CRB-G Action Group}, has been formed among Pacific-based entomologists working on the CRB-G problem. Participants from Guam, Hawaii, Palau, Papua New Guinea, Solomon Islands, Fiji, Malaysia, Japan and New Zealand have met several times and future meetings are planned (Table \ref{tab:action-group}).  This is an \textit{ad hoc} group which has been organized by Dr. Trevor Jackson and Sean Marshall of AgResearch New Zealand. AgResearch is recognized as a global center for expertise on biological control of CRB. AgResearch scientists have worked on CRB in the south Pacific for several decades and they maintain a library of OrNV isolates in cell culture. The New Zealand government has recently committed several million dollars to aid response to CRB-G in the south Pacific islands. Although individual institutions working to find a solution to the CRB-G problem on American-affiliated islands in the northern Pacific have secured funding from multiple, short-term grants, attempts to secure funding to support a sustainable well-coordinated regional project have been unsuccessful. 

\begin{table}[h]
	\centering
	\caption{Meetings of the CRB-G Action Group}
	\begin{tabular}{l}
		\toprule
		2015 Pacific Entomology Conference, Honolulu, HI, USA \\
		2016 International Congress of Entomology, Orlando, USA \\
		2017 Japanese Society for Insect Pathology, Tokyo, Japan \\
		2018 Society for Invertebrate Pathology, Gold Coast, Australia \\
		2019 XIX International Plant Protection Congress, Hyderabad, India \\
		2020 (tentative): Pacific Plant Protection Organization, Guam \\
		\bottomrule
	\end{tabular}
	\label{tab:action-group}
\end{table}	

\subsection{Participation in Scientific Meetings}

Moore and Grasela participated at the XIX International Plant Protection Congress in a symposium entitled \textit{The challenge of coconut rhinoceros beetle, Oryctes rhinoceros, to palm production and prospects for control in a changing world}. Moore made an oral presentation at this meeting \parencite{moore_status_2019}.
They also participated in a CRB-G Action Group meeting with colleagues from throughout the Pacific and Asia.

\subsection{Development of Online Resources}

Project resources were used to build and maintain the following:
\begin{itemize}
	\item CRB Bibliography \url{https://github.com/aubreymoore/CRB-Bibliography}
	\item Interactive CRB Invasion History Map	\url{https://aubreymoore.github.io/crbdist/mymap.html}
	\item CRB Wiki Site \url{https://guaminsects.net/CRBG}
	\item CRB-G Facebook Site \url{https://www.facebook.com/groups/crbg07} 
\end{itemize}

\newpage

%\section{Appendices}
%
%\appendix
%
%\section{\label{sec:Appendix-A}Appendix A: Technical Report: Injection Bioassay
%of OrNV Isolate DUG42}
%
%See following page.
%
%%\includepdf[pages=-]{DUG42}
%
%\section{\label{sec:Appendix-B}Appendix B: Technical Report: Injection Bioassay
%of OrNV Isolate MALB}
%
%See following page.
%
%%\includepdf[pages=-]{MALB}
%
%\section{\label{sec:Appendix-C}Appendix C: Technical Report: Injection Bioassay
%of OrNV Isolate PNG}
%
%See following page.
%
%%\includepdf[pages=-]{PNG}
%
%
%\section{\label{sec:Appendix-D}Appendix D: Technical Report: Injection Bioassay
%of OrNV Isolate V23B}
%
%See following page.
%
%%\includepdf[pages=-]{V23B}
\clearpage
\printbibliography[heading=bibintoc]


\end{document}
